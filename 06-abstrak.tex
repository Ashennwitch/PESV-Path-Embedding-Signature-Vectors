%
% Halaman Abstrak
%
% @author  Andreas Febrian
% @version 1.00
%
%\chapter*{Abstrak}
\setstretch{1}
\vspace*{0.2cm}
%    \begingroup
    \singlespacing
	\setlength{\parindent}{0pt}
	
	\begin{tabular}{@{}l l p{10cm}}
		Nama&: & \penulis \\
		Program Studi&: & \program \\
		Judul&: & \judul \\
		Pembimbing&: & \pembimbing \\
	\end{tabular}

	\bigskip
	\bigskip

Adopsi \textit{Virtual Private Network} (VPN) yang masif telah meningkatkan privasi pengguna, namun di sisi lain menimbulkan tantangan signifikan bagi manajemen dan keamanan jaringan dengan menyamarkan konten lalu lintas data. Metode analisis tradisional seperti \textit{Deep Packet Inspection} (DPI) menjadi tidak efektif akibat enkripsi, sehingga menciptakan kebutuhan krusial akan teknik yang mampu mengklasifikasikan lalu lintas tanpa memeriksa \textit{payload}-nya. Meskipun banyak metode dapat mendeteksi keberadaan VPN, klasifikasi rinci (\textit{fine-grained}) terhadap aplikasi atau kategori spesifik di dalam terowongan terenkripsi masih menjadi masalah yang kompleks. Penelitian ini memperkenalkan sebuah kerangka kerja baru berbasis aliran (\textit{flow}), yaitu \textit{Path-Embedding Signature Vector} (PESV), yang dirancang untuk mengkarakterisasi dan mengklasifikasikan lalu lintas VPN terenkripsi berdasarkan dinamika perilaku tingkat tingginya. PESV yang diusulkan merupakan sebuah vektor signature multi-aspek, $\Sigma = (\alpha, \beta, \gamma)$, di mana setiap komponen menangkap aspek yang berbeda dari perilaku aliran lalu lintas. Komponen pertama, $\alpha$, memanfaatkan \textit{autoencoder} berbasis sekuens untuk menghasilkan \textit{embedding} dari urutan ukuran paket, yang menangkap pola struktural pertukaran data. Komponen kedua, $\beta$, menggunakan jarak Wasserstein untuk mengukur ketidaksamaan antara distribusi waktu antar-kedatangan (\textit{interarrival time}) paket, yang secara efektif memprofilkan ritme dan periodisitas temporal aliran. Komponen terakhir, $\gamma$, menghitung similaritas kosinus dari statistik \textit{burst-level} untuk mengkarakterisasi dinamika makro komunikasi. Metode ini dievaluasi menggunakan dataset ISCX 2016, yang telah melalui pra-pemrosesan menghasilkan 8.763 aliran lalu lintas tervalidasi. Vektor PESV yang dihasilkan membentuk sebuah basis data kaya fitur yang digunakan untuk melatih dan mengevaluasi sebuah pengklasifikasi \textit{machine learning} untuk identifikasi tingkat aplikasi dan kategori. Kerangka kerja yang diusulkan ini menunjukkan bahwa dengan merepresentasikan aliran lalu lintas melalui perilaku intrinsiknya—seperti urutan, waktu, dan \textit{burstiness}—kinerja klasifikasi yang tinggi dapat dicapai, sehingga menawarkan solusi yang kuat untuk memperoleh visibilitas kritis ke dalam lalu lintas jaringan terenkripsi.

%[\catatan{Sesuaikan ia abstrak berdasarkan ketentuan berikut}: Abstrak merupakan ikhtisar padat dan informatif dari keseluruhan tugas akhir yang esensinya \catatan{memuat permasalahan}, \catatan{tujuan}, \catatan{metode penelitian}, \catatan{hasil}, dan \catatan{kesimpulan}, dirancang untuk memudahkan pembaca memahami secara cepat isi tugas akhir dan memutuskan relevansinya. Abstrak ini harus \catatan{ditulis dalam satu paragraf tunggal} dengan \catatan{panjang maksimum 500 kata}, menggunakan \catatan{font Times New Roman} \catatan{ukuran 12 poin} dengan \catatan{spasi tunggal}. Penting untuk dicatat bahwa abstrak harus disusun dalam dua bahasa, yaitu \catatan{Bahasa Indonesia} dan \catatan{Bahasa Inggris}, di mana setiap versi bahasa mengikuti ketentuan format yang sama dan sedapat mungkin diletakkan dalam satu halaman. Di bagian atas abstrak, harus dicantumkan \catatan{Nama Mahasiswa (tanpa NPM)}, \catatan{Program Studi}, dan \catatan{Judul Tugas Akhir}. Bagian bawah setiap abstrak (baik Bahasa Indonesia maupun Bahasa Inggris) harus diikuti oleh \catatan{Kata Kunci} yang relevan, disajikan dalam bahasa yang sesuai (Bahasa Indonesia untuk abstrak Bahasa Indonesia, dan padanan Bahasa Inggris untuk abstrak Bahasa Inggris). Semua istilah asing, kecuali nama, \catatan{wajib dicetak miring (italic)}. Isi spesifik abstrak dapat disesuaikan dengan keilmuan masing-masing bidang studi.]
	\bigskip

	Kata kunci:	klasifikasi lalu lintas VPN, analisis lalu lintas terenkripsi, klasifikasi berbasis aliran, \textit{machine learning}, \textit{deep learning}, \textit{autoencoder}, vektor signature, ISCX 2016.
%    \endgroup

\newpage