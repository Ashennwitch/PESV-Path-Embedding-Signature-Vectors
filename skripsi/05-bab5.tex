%-----------------------------------------------------------------------------%
\chapter{\babLima}
\label{cha:penutup}

Bab ini merupakan bagian akhir dari laporan tugas akhir yang merangkum keseluruhan hasil penelitian dan pembahasan yang telah diuraikan pada bab-bab sebelumnya. 
Bagian pertama, Kesimpulan, akan menjawab secara lugas setiap butir rumusan masalah yang telah ditetapkan. 
Bagian kedua, Saran, akan memberikan rekomendasi untuk pengembangan sistem D'Office dan templat di masa mendatang, serta arahan untuk penelitian selanjutnya yang relevan.

\section{Kesimpulan}
\label{sec:kesimpulan}
Berdasarkan hasil perancangan, implementasi, dan evaluasi yang telah dilakukan, dapat ditarik beberapa kesimpulan utama yang selaras dengan tujuan penelitian:
\begin{enumerate}
    \item Telah berhasil dirancang dan dibangun sebuah purwarupa fungsional sistem pengelolaan tugas akhir berbasis web bernama D'Office. 
    Sistem ini berhasil mengintegrasikan dan menyederhanakan alur kerja administrasi, mulai dari pengajuan judul hingga pengunggahan berkas final, yang sebelumnya berjalan secara manual dan terdesentralisasi.
    
    \item Sistem D'Office terbukti secara kuantitatif dapat meningkatkan efektivitas dan efisiensi proses administrasi tugas akhir. 
    Hasil pengujian menunjukkan bahwa penggunaan sistem ini berhasil mengurangi Waktu Pengerjaan Tugas (\textit{Time on Task}) rata-rata sebesar 76\% dan meningkatkan Tingkat Keberhasilan Tugas (\textit{Task Success Rate}) hingga 100\% pada skenario-skenario kunci. 
    Kinerja teknis sistem dari sisi basis data dan jaringan juga terbukti sangat responsif.
    
    \item Templat \LaTeX{} yang dikembangkan sebagai bagian dari solusi terpadu terbukti sangat efektif dalam meningkatkan akurasi dan konsistensi format penulisan. 
    Kelompok pengguna yang menggunakan templat berhasil mencapai Skor Kepatuhan Format rata-rata 97.5 dari 100, jauh lebih tinggi dibandingkan kelompok kontrol (74.0). 
    Hal ini memvalidasi dampak positif dari templat dalam mengurangi revisi non-substantif.
    
    \item Secara kualitatif, antarmuka pengguna (UI) dan pengalaman pengguna (UX) dari sistem D'Office dinilai sangat baik. 
    Analisis heuristik menunjukkan bahwa desain sistem telah menerapkan prinsip-prinsip kebergunaan yang fundamental, seperti visibilitas status sistem, konsistensi, dan kontrol pengguna. 
    Hal ini juga didukung oleh skor \ac{SUS} sebesar 85.5, yang masuk dalam kategori \textit{Excellent}.
\end{enumerate}

\section{Saran}
\label{sec:saran}
Berdasarkan temuan dan keterbatasan dalam penelitian ini, berikut adalah beberapa saran yang dapat menjadi pertimbangan untuk pengembangan dan penelitian di masa mendatang:

\begin{enumerate}
    \item \textbf{Untuk Pengembangan Sistem D'Office:}
    \begin{itemize}
        \item Prioritas utama pengembangan selanjutnya adalah membangun jembatan integrasi (\textit{API bridge}) secara \textit{real-time} dengan Sistem Informasi Akademik (SIAK-NG) untuk mengotomatiskan sinkronisasi data mahasiswa dan catatan bimbingan.
        \item Menambahkan modul analisis anti-plagiarisme yang terintegrasi langsung di dalam sistem, sehingga dosen dapat memeriksa orisinalitas naskah tanpa perlu menggunakan platform eksternal.
    \end{itemize}
    
    \item \textbf{Untuk Penelitian Selanjutnya:}
    \begin{itemize}
        \item Melakukan studi longitudinal dengan menerapkan sistem D'Office dan templat pada satu angkatan penuh selama satu tahun akademik. 
        Penelitian semacam ini akan memberikan data yang lebih kaya mengenai dampak jangka panjang terhadap waktu kelulusan dan kualitas tugas akhir.
        \item Mengembangkan dan mengevaluasi efektivitas modul-modul tambahan, seperti sistem penjadwalan ruang sidang otomatis atau forum diskusi terintegrasi antara mahasiswa bimbingan dengan dosennya.
    \end{itemize}
\end{enumerate}