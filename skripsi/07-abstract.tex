%
% Halaman Abstract
%
% @author  Andreas Febrian
% @version 1.00
%

%\chapter*{Abstract}
\setstretch{1}
\vspace*{0.2cm}
{
%    \begingroup
    \singlespacing
	\setlength{\parindent}{0pt}
	
	\begin{tabular}{@{}l l p{10cm}}
		Name&: & \penulis \\
		Study Program&: & \program \\
		Title&: & \judulInggris \\
		Supervisor&: & \pembimbing \\
	\end{tabular}

	\bigskip
	\bigskip
    
The classification of encrypted network traffic is a critical challenge for network management and cybersecurity. This research proposes and validates a novel, flow-based Hybrid Flow Vector (HFV), designed to classify both encrypted (VPN) and non-encrypted (Non-VPN) traffic from the ISCX 2016 dataset. This model demonstrates the superiority of a hybrid approach that combines deep learning with robust statistical feature engineering.

The HFV is a multi-modal vector composed of three components: ($\alpha$) a 128-dimension feature set extracted from a 1D-Convolutional Neural Network (1D-CNN) trained on raw packet payloads; ($\beta$) a 39-feature set of comprehensive flow-level statistics; and ($\gamma$) a 37-feature set detailing burst-level statistics.

An extensive ablation study confirmed that the deep learning and statistical features are highly complementary. The full hybrid model ($\alpha + \beta + \gamma$) achieved the highest accuracy for both 6-class category (81.40\%) and 6-class application (76.58\%) classification, significantly outperforming both the deep-learning-only model (77.48\%) and the statistical-only model (75.43\%). A final classifier comparison showed that an XGBoost model, trained on the complete HFV, yielded the best performance, achieving peak accuracies of 97.12\% for binary (VPN/Non-VPN) and 81.51\% for category classification. This work proves that a hybrid model, integrating byte-level patterns from 1D-CNNs with flow-level statistical analysis, provides a highly robust and accurate solution for complex traffic classification.

%The abstract is a concise and informative summary of the entire final project, essentially \catatan{covering the problem}, \catatan{objectives}, \catatan{research methods}, \catatan{results}, and \catatan{conclusions}, designed to quickly help readers understand the content of the final project and decide its relevance. This abstract must be \catatan{written in a single paragraph} with a \catatan{maximum length of 500 words}, using \catatan{Times New Roman font} \catatan{12 points}, with \catatan{single line spacing}. It is important to note that the abstract must be prepared in two languages, namely \catatan{Indonesian} and \catatan{English}, where each language version follows the same format provisions and, if possible, should be placed on one page. At the top of the abstract, the \catatan{Student's Name (without student ID)}, \catatan{Study Program}, and \catatan{Final Project Title} must be included. The bottom of each abstract (both Indonesian and English versions) must be followed by \catatan{relevant Keywords}, presented in the appropriate language (Indonesian for the Indonesian abstract, and English equivalents for the English abstract). The specific content of the abstract can be adjusted according to the respective field of study.
	\bigskip

	Keywords:
	VPN traffic classification, encrypted traffic analysis, flow-based classification, machine learning, deep learning, autoencoder, signature vector, ISCX 2016.
%    \endgroup
}

\newpage