%
% Halaman Abstrak
%
% @author  Andreas Febrian
% @version 1.00
%
%\chapter*{Abstrak}
\setstretch{1}
\vspace*{0.2cm}
%    \begingroup
    \singlespacing
	\setlength{\parindent}{0pt}
	
	\begin{tabular}{@{}l l p{10cm}}
		Nama&: & \penulis \\
		Program Studi&: & \program \\
		Judul&: & \judul \\
		Pembimbing&: & \pembimbing \\
	\end{tabular}

	\bigskip
	\bigskip

Klasifikasi lalu lintas jaringan terenkripsi merupakan tantangan penting untuk manajemen jaringan dan keamanan siber. Penelitian ini mengusulkan dan memvalidasi sebuah model baru berbasis \textit{flow}, Hybrid Flow Vector (HFV), yang dirancang untuk mengklasifikasikan lalu lintas terenkripsi (VPN) dan non-enkripsi (Non-VPN) dari dataset ISCX 2016. Model ini menunjukkan keunggulan pendekatan hibrida yang menggabungkan deep learning dengan rekayasa fitur statistik.

HFV adalah vektor multi-modal yang terdiri dari tiga komponen: ($\alpha$) satu set fitur 128-dimensi yang diekstraksi dari 1D-Convolutional Neural Network (1D-CNN) yang dilatih pada payload paket mentah; ($\beta$) satu set fitur 39-dimensi berisi statistik level aliran (flow) yang komprehensif; dan ($\gamma$) satu set fitur 37-dimensi yang merinci statistik level burst.

Studi ablasi ekstensif mengkonfirmasi bahwa fitur deep learning dan fitur statistik bersifat sangat komplementer. Model hibrida penuh ($\alpha + \beta + \gamma$) mencapai akurasi tertinggi untuk klasifikasi kategori 6-kelas (81.40\%) dan aplikasi 6-kelas (76.58\%), secara signifikan mengungguli model deep-learning-saja (77.48\%) maupun model statistik-saja (75.43\%). Perbandingan klasifikasi akhir menunjukkan bahwa model XGBoost, yang dilatih pada HFV, memberikan performa terbaik, mencapai akurasi puncak 97.12\% untuk klasifikasi biner (VPN/Non-VPN) dan 81.51\% untuk klasifikasi kategori. Penelitian ini membuktikan bahwa model hibrida, yang mengintegrasikan pola level-byte dari 1D-CNN dengan analisis statistik level-flow, memberikan solusi yang tangguh dan akurat untuk klasifikasi lalu lintas yang kompleks.

%[\catatan{Sesuaikan ia abstrak berdasarkan ketentuan berikut}: Abstrak merupakan ikhtisar padat dan informatif dari keseluruhan tugas akhir yang esensinya \catatan{memuat permasalahan}, \catatan{tujuan}, \catatan{metode penelitian}, \catatan{hasil}, dan \catatan{kesimpulan}, dirancang untuk memudahkan pembaca memahami secara cepat isi tugas akhir dan memutuskan relevansinya. Abstrak ini harus \catatan{ditulis dalam satu paragraf tunggal} dengan \catatan{panjang maksimum 500 kata}, menggunakan \catatan{font Times New Roman} \catatan{ukuran 12 poin} dengan \catatan{spasi tunggal}. Penting untuk dicatat bahwa abstrak harus disusun dalam dua bahasa, yaitu \catatan{Bahasa Indonesia} dan \catatan{Bahasa Inggris}, di mana setiap versi bahasa mengikuti ketentuan format yang sama dan sedapat mungkin diletakkan dalam satu halaman. Di bagian atas abstrak, harus dicantumkan \catatan{Nama Mahasiswa (tanpa NPM)}, \catatan{Program Studi}, dan \catatan{Judul Tugas Akhir}. Bagian bawah setiap abstrak (baik Bahasa Indonesia maupun Bahasa Inggris) harus diikuti oleh \catatan{Kata Kunci} yang relevan, disajikan dalam bahasa yang sesuai (Bahasa Indonesia untuk abstrak Bahasa Indonesia, dan padanan Bahasa Inggris untuk abstrak Bahasa Inggris). Semua istilah asing, kecuali nama, \catatan{wajib dicetak miring (italic)}. Isi spesifik abstrak dapat disesuaikan dengan keilmuan masing-masing bidang studi.]
	\bigskip

	Kata kunci:	klasifikasi lalu lintas VPN, analisis lalu lintas terenkripsi, klasifikasi berbasis aliran, \textit{machine learning}, \textit{deep learning}, \textit{autoencoder}, vektor signature, ISCX 2016.
%    \endgroup

\newpage