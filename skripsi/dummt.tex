Klasifikasi lalu lintas jaringan terenkripsi merupakan tantangan penting untuk manajemen jaringan dan keamanan siber. Penelitian ini mengusulkan dan memvalidasi sebuah model baru berbasis aliran (flow), \textit{Hybrid Flow Vector} (HFV), yang dirancang untuk mengklasifikasikan lalu lintas terenkripsi (VPN) dan non-enkripsi (Non-VPN) dari dataset ISCX 2016. HFV adalah vektor multi-modal yang terdiri dari tiga komponen: ($\alpha$) satu set fitur 128-dimensi yang diekstraksi dari 1D-Convolutional Neural Network (1D-CNN) pada \textit{payload} paket mentah; ($\beta$) satu set fitur 39-dimensi berisi statistik level aliran (\textit{flow}); dan ($\gamma$) satu set fitur 37-dimensi yang merinci statistik level \textit{burst}.

Studi ablasi pada dataset gabungan (VPN/Non-VPN) memberikan justifikasi untuk utilisasi model penuh ($\alpha$ + $\beta$ + $\gamma$). Model hibrida penuh ini secara konsisten mencapai akurasi tertinggi di semua tugas klasifikasi, termasuk akurasi puncak 97.01\% untuk klasifikasi biner dan 81.40\% untuk klasifikasi kategori, membuktikan bahwa fitur \textit{deep learning} dan statistik bersifat sangat komplementer.

Selanjutnya, dengan berfokus pada masalah inti klasifikasi \textit{di dalam} lalu lintas VPN, analisis mengungkap wawasan yang lebih bernuansa. Untuk klasifikasi kategori VPN, model hibrida ($\alpha$ + $\gamma$) mencapai performa optimal 94.48\%, mengungguli model \textit{deep-learning-saja} (90.48%) dan \textit{statistik-saja} (93.33%). Menariknya, untuk klasifikasi aplikasi VPN, model \textit{statistik-saja} ($\beta$ + $\gamma$) terbukti paling akurat dengan 91.43\%. Penelitian ini membuktikan bahwa kerangka kerja HFV menyediakan solusi yang tangguh, di mana model hibrida penuh efektif untuk klasifikasi umum, namun set fitur optimal dapat disesuaikan bergantung pada granularitas tugas klasifikasi terenkripsi yang dituju.

Classifying encrypted network traffic is a critical challenge for network management and cybersecurity. This research proposes and validates a novel flow-based model, the \textit{Hybrid Flow Vector} (HFV), designed to classify encrypted (VPN) and non-encrypted (Non-VPN) traffic from the ISCX 2016 dataset. The HFV is a multi-modal vector comprising three components: ($\alpha$) a 128-dimension feature set extracted from a 1D-Convolutional Neural Network (1D-CNN) on raw packet payloads; ($\beta$) a 39-dimension feature set of flow-level statistics; and ($\gamma$) a 37-dimension feature set detailing burst-level statistics.

An ablation study on the combined (VPN/Non-VPN) dataset justifies the utilization of the full model ($\alpha$ + $\beta$ + $\gamma$). This full hybrid model consistently achieved the highest accuracy across all classification tasks, including a peak accuracy of 97.01\% for binary classification and 81.40\% for category classification, proving that the deep learning and statistical features are highly complementary.

Furthermore, by focusing on the core challenge of classifying \textit{within} VPN traffic, the analysis revealed a more nuanced insight. For VPN category classification, the hybrid ($\alpha$ + $\gamma$) model achieved optimal performance at 94.48\%, outperforming deep-learning-only (90.48\%) and statistical-only (93.33\%) models. Interestingly, for VPN application classification, the statistical-only ($\beta$ + $\gamma$) model proved most accurate at 91.43\%. This research demonstrates that the HFV framework provides a robust solution, where the full hybrid model is effective for general classification, yet the optimal feature set can be fine-tuned depending on the granularity of the specific encrypted classification task.