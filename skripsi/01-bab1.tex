%-----------------------------------------------------------------------------%
\chapter{\babSatu}
%-----------------------------------------------------------------------------%

%-----------------------------------------------------------------------------%
\section{Latar Belakang}
\label{sec:latar_belakang}
%-----------------------------------------------------------------------------%
Transformasi digital telah mengubah cara organisasi beroperasi dan cara individu berkomunikasi. Fenomena ini diiringi dengan peningkatan kesadaran akan privasi dan keamanan data, yang mendorong adopsi enkripsi secara masif. Saat ini, enkripsi telah menjadi standar de-facto. Data terbaru per September 2025 menunjukkan bahwa sekitar 95\% lalu lintas web telah dienkripsi menggunakan protokol HTTPS \citep{googleGoogleTransparency}.

Meskipun sangat fundamental untuk privasi, adopsi enkripsi dan VPN secara masif ini menghadirkan tantangan besar bagi manajemen dan keamanan jaringan \citep{wright2004hmm}. Model keamanan tradisional seperti Deep Packet Inspection (DPI), yang bergantung pada kemampuan untuk memeriksa konten (payload) paket data, kini menjadi "buta" \citep{shapira2021flowpic, anderson2018deciphering}.

Hal ini menciptakan kesenjangan (gap) visibilitas yang kritis, yang secara aktif dieksploitasi oleh pelaku kejahatan siber. Pelaku kejahatan siber semakin memanfaatkan saluran terenkripsi untuk menyembunyikan aktivitas jahat \citep{wright2004hmm}. Sebuah laporan dari Zscaler ThreatLabz pada tahun 2023 mengungkapkan bahwa 85.9\% serangan siber kini menggunakan saluran terenkripsi, menandai peningkatan 20\% dari tahun sebelumnya \citep{zscalerThreatLabz2023}. Laporan Zscaler (2025) menyoroti risiko nyata ini, menemukan bahwa 56\% organisasi telah mengalami serangan siber yang terkait langsung dengan kerentanan VPN hanya dalam satu tahun terakhir \citep{zscalerZscalerThreatLabz}.

Risiko ini bukan lagi teoretis; 92\% responden survei Zscaler (2025) menyatakan kekhawatiran bahwa kerentanan VPN yang tidak segera ditambal akan secara langsung memicu serangan ransomware \citep{zscalerZscalerThreatLabz}. Pelaku kejahatan siber menggunakan enkripsi VPN untuk menyembunyikan aktivitas berbahaya, seperti gerakan lateral di dalam jaringan (sebuah taktik yang menjadi kekhawatiran utama bagi 71\% responden \citep{zscalerZscalerThreatLabz}) dan eksfiltrasi data sensitif \citep{zscalerZscalerThreatLabz}.

Upaya untuk mengatasi masalah ini terbagi dalam dua pendekatan utama, yang keduanya memiliki kelemahan signifikan. Pendekatan pertama adalah dekripsi SSL/TLS (sering disebut Man-in-the-Middle), di mana lalu lintas didekripsi di gateway keamanan untuk diinspeksi. Namun, metode ini sangat mahal secara komputasi, menimbulkan masalah kepatuhan dan privasi yang serius, serta seringkali gagal jika aplikasi menggunakan teknik seperti \textit{certificate pinning}. Solusi yang ideal adalah yang mampu melakukan pemantauan jaringan dan deteksi anomali secara efektif tanpa mendekripsi data, sehingga menjaga privasi pengguna \citep{lopez2017network}.

Pendekatan kedua, yang menjadi fokus dunia akademis, adalah Encrypted Traffic Analysis (ETA) \citep{wright2004hmm}. Teori dasarnya adalah bahwa meskipun konten paket dienkripsi, metadata dan pola perilaku dari arus lalu lintas (flow) tetap dapat diamati \citep{noauthor_2025-nh}. Pola-pola ini—seperti urutan ukuran paket, waktu antar-kedatangan paket (IAT), dan volume data—dapat menciptakan "sidik jari" (\textit{fingerprint}) unik untuk setiap jenis aplikasi atau aktivitas \citep{Yao2022-lu}. Sebuah survei komprehensif oleh Razooqi \& Pekar (2025) mengonfirmasi bahwa analisis lalu lintas VPN telah menjadi bidang penelitian kritis \citep{Razooqi2025-bo, shahbar2015traffic}, dengan fokus pada tiga tugas utama: deteksi VPN, identifikasi VPN spesifik, dan klasifikasi aplikasi di dalam VPN \citep{Razooqi2025-bo}.

Kajian empiris terbaru telah membuktikan validitas pendekatan ETA. Penelitian telah menunjukkan bahwa model Deep Learning efektif dalam menganalisis data sekuensial mentah \citep{rezaei2018deep}. Sebagai contoh, Tawfeeq et al. (2025) sukses menerapkan model EfficientNet dan BiLSTM pada urutan paket mentah  untuk klasifikasi \citep{noauthor_2025-nh}, sementara Yao et al. (2022) menggunakan LSTM dengan mekanisme \textit{attention} pada data \textit{time-series} lalu lintas \citep{Yao2022-lu}. Pendekatan lain berfokus pada fitur statistik dan temporal; Shapira et al. (2021) secara inovatif mengubah data ukuran paket dan IAT menjadi representasi gambar "FlowPic" dan mengklasifikasikannya menggunakan CNN \citep{shapira2021flowpic}. Konsep mengubah lalu lintas menjadi gambar untuk dianalisis oleh CNN juga telah dibuktikan dalam penelitian lain \citep{he2020image}. Kotak et al. (2025) juga menunjukkan keberhasilan menggunakan fitur statistik berbasis \textit{time-series} untuk klasifikasi \citep{kotak2025vpn}. Tren yang lebih maju bahkan memanfaatkan autoencoder untuk mempelajari representasi fitur secara otomatis \citep{cui2024attention, lv2023aae} atau menggunakan model bahasa canggih seperti BERT untuk klasifikasi multi-task \citep{park2024fast, lin2022bert}.

Meskipun demikian, sebagian besar penelitian ini cenderung berfokus pada satu jenis fitur saja \citep{velan2015survey, aminuddin2018survey}. Kelemahannya adalah fitur-fitur ini, jika digunakan secara terisolasi, mungkin tidak cukup \textit{robust} untuk menghadapi variasi lalu lintas di dunia nyata \citep{aceto2021distiller} atau teknik \textit{obfuscation} yang semakin canggih \citep{sharma2024survey}. Kesenjangan yang ada saat ini \citep{sharma2024survey} adalah belum adanya penelitian yang mengintegrasikan tiga pilar analisis perilaku—struktur sekuensial, distribusi temporal, dan dinamika \textit{burst}—ke dalam satu vektor tanda tangan (\textit{signature vector}) yang kohesif menggunakan kombinasi teknik pemodelan canggih. Ditambah, walaupun penelitian di bidang Encrypted Traffic Analysis (ETA) telah banyak mengeksplorasi deteksi biner (VPN vs Non-VPN), kemampuan untuk melakukan identifikasi fine-grained—yaitu membedakan aplikasi spesifik (misalnya, YouTube) atau kategori aplikasi (misalnya, Streaming) di dalam terowongan VPN—masih menjadi masalah yang kompleks dan terbuka. Urgensi untuk mengembangkan metode yang lebih \textit{robust} sangat tinggi, karena kegagalan mendeteksi lalu lintas berbahaya secara akurat dapat berakibat pada pelanggaran data, penyebaran \textit{ransomware} (seperti yang ditakutkan 92\% organisasi \citep{zscalerZscalerThreatLabz}), dan hilangnya kendali atas infrastruktur jaringan.

Oleh karena itu, penelitian ini akan fokus pada pengembangan sebuah model representasi baru yang disebut Path-Embedding Signature Vector (PESV) $\Sigma = (\alpha, \beta, \gamma)$. Berdasarkan dataset 8.763 arus data bersih yang telah disiapkan, penelitian ini mengusulkan:
\begin{itemize}
    \item Komponen $\alpha$: Representasi sekuensial yang dipelajari menggunakan LSTM Autoencoder pada urutan ukuran paket.
    \item Komponen $\beta$: Representasi distribusi temporal menggunakan Wasserstein Distance dari distribusi IAT.
    \item Komponen $\gamma$: Representasi dinamika \textit{burst} menggunakan Cosine Similarity dari statistik \textit{burst}.
\end{itemize}

Penelitian ini bermaksud untuk membuktikan bahwa PESV, dengan menangkap perilaku arus data secara holistik dari tiga perspektif yang saling melengkapi, mampu menghasilkan model klasifikasi lalu lintas terenkripsi yang lebih akurat dan \textit{robust} dibandingkan dengan metode yang ada saat ini.



%-----------------------------------------------------------------------------%
\section{Rumusan Masalah}
\label{sec:rumusan_masalah}
%-----------------------------------------------------------------------------%
Untuk menjembatani kesenjangan dan menjawab tantangan seperti yang telah diuraikan di latar belakang, penelitian ini difokuskan untuk menjawab pertanyaan-pertanyaan berikut:
\begin{enumerate}
    \item Bagaimana cara mengembangkan suatu kerangka kerja klasifikasi yang mampu mengidentifikasi secara granular, baik aplikasi spesifik (misalnya, YouTube, Skype) maupun kategori aplikasi (misalnya, Streaming, VoIP), di dalam aliran lalu lintas VPN yang telah terenkripsi penuh tanpa bergantung pada inspeksi payload?
    
    \item Bagaimana merancang dan mengimplementasikan sebuah model representasi fitur multidimensional, Path-Embedding Signature Vector (PESV) $\Sigma=(\alpha,\beta,\gamma)$, yang secara kohesif mengintegrasikan tiga pilar analisis perilaku: (a) pola struktural pertukaran data, (b) ritme dan periodisitas temporal aliran, serta (c) dinamika makro komunikasi (burstiness)?
    
    \item Seberapa tinggi tingkat akurasi dan robustisitas yang dapat dicapai oleh model klasifikasi machine learning yang dilatih menggunakan representasi holistik PESV, dan apakah pendekatan ini terbukti lebih unggul dibandingkan dengan metode yang hanya mengandalkan satu dimensi fitur saja dalam mengklasifikasikan lalu lintas pada dataset ISCX 2016?
\end{enumerate}

%-----------------------------------------------------------------------------%
\section{Tujuan Penelitian}
\label{sec:tujuan_penelitian}
%-----------------------------------------------------------------------------%
Selaras dengan rumusan masalah yang telah dipaparkan, tujuan dari penelitian ini adalah sebagai berikut:
\begin{enumerate}
    \item Membangun sebuah kerangka kerja yang mampu melakukan klasifikasi (\textit{fine-grained}) (identifikasi aplikasi dan kategori aplikasi) terhadap lalu lintas VPN terenkripsi , dengan mengandalkan analisis perilaku aliran data tanpa melakukan inspeksi payload.
    
    \item Merancang dan mengimplementasikan sebuah model representasi fitur holistik, Path-Embedding Signature Vector (PESV) $\Sigma=(\alpha,\beta,\gamma)$, yang secara kohesif mengintegrasikan tiga pilar analisis perilaku aliran data.

    \item Mengevaluasi dan membuktikan secara kuantitatif bahwa kerangka kerja PESV yang diusulkan mampu mencapai kinerja klasifikasi yang tinggi, akurat, dan robust dalam mengidentifikasi aplikasi dan kategori di dalam lalu lintas terenkripsi, melebihi pendekatan yang hanya mengandalkan satu dimensi fitur secara terisolasi.
\end{enumerate}

%-----------------------------------------------------------------------------%
\section{Manfaat Penelitian}
\label{sec:manfaat_penelitian}
%-----------------------------------------------------------------------------%
Berdasarkan latar belakang, rumusan masalah, dan tujuan penelitian yang telah diuraikan, penelitian ini diharapkan memberikan manfaat sebagai berikut:

Manfaat Teoritis:

\begin{enumerate}
    \item Mengenalkan Path-Embedding Signature Vector (PESV) $\Sigma=(\alpha,\beta,\gamma)$ sebagai sebuah kerangka kerja representasi fitur holistik baru dalam domain Analisis Lalu Lintas Terenkripsi.

    \item Memberikan validasi empiris mengenai efektivitas penggabungan deep learning (Autoencoder) dengan metrik statistik (Wasserstein Distance, Cosine Similarity) untuk feature engineering hibrida.

    \item Menyediakan hasil tolok ukur (benchmark) baru untuk perbandingan kinerja metode klasifikasi fine-grained pada dataset ISCX 2016 bagi peneliti selanjutnya.
\end{enumerate}

Manfaat Praktis:

\begin{enumerate}
    \item Memberikan administrator jaringan solusi payload-agnostic untuk memulihkan visibilitas guna penerapan Quality of Service (QoS) yang lebih efektif.

    \item Menyediakan alat bantu bagi praktisi keamanan siber untuk mendeteksi anomali dan aktivitas mencurigakan di dalam saluran VPN tanpa dekripsi.

    \item Membantu organisasi menegakkan tata kelola kebijakan jaringan dan keamanan secara efisien tanpa mengorbankan privasi pengguna.
\end{enumerate}

%-----------------------------------------------------------------------------%
\section{Batasan Penelitian}
\label{sec:ruang_lingkup}
%-----------------------------------------------------------------------------%
Untuk menjaga fokus dan kelayakan studi, terdapat beberapa batasan yang perlu diperhatikan dalam menginterpretasi hasil penelitian:
\begin{enumerate}
    \item Generalisasi model terbatas pada karakteristik dataset ISCX 2016, yang mungkin belum sepenuhnya merepresentasikan pola lalu lintas aplikasi VPN modern.
    
    \item Proses textit{pre-processing} sedemikian rupa menghasilkan 8.763 aliran (textit{flow}) data ideal, sehingga cakupan model tidak mencakup analisis terhadap aliran data terfragmentasi atau tidak lengkap.
    
    \item Ruang lingkup \textit{feature engineering} terfokus secara eksklusif pada tiga pilar perilaku yang didefinisikan dalam kerangka Path-Embedding Signature Vector (PESV) $\Sigma=(\alpha,\beta,\gamma)$.
    
    \item Kinerja komponen $\beta$ dan $\gamma$ bergantung pada kualitas prototipe yang telah ditentukan sebelumnya, di mana metode optimasi pembuatan prototipe tersebut berada di luar cakupan studi ini.
\end{enumerate}

%-----------------------------------------------------------------------------%
\section{Sistematika Penulisan}
\label{sec:sistematika}
%-----------------------------------------------------------------------------%
Laporan \ac{TA} ini disusun secara sistematis ke dalam lima bab utama yang saling berkaitan, dengan rincian sebagai berikut:

\begin{enumerate}
    \item \textbf{Bab 1: Pendahuluan.} 
    Bab ini menyajikan gambaran umum penelitian. Bagian ini diawali dengan latar belakang yang menjelaskan urgensi masalah akibat masifnya adopsi enkripsi dan VPN , yang menyebabkan metode analisis tradisional seperti DPI menjadi tidak efektif. Bab ini juga memaparkan rumusan masalah yang berfokus pada kesenjangan (gap) dalam klasifikasi fine-grained, serta menjabarkan tujuan penelitian, manfaat (teoritis dan praktis), batasan penelitian, dan diakhiri dengan sistematika penulisan laporan ini.
    
    \item \textbf{Bab 2: Tinjauan Pustaka dan Landasan Teori.} 
    Bab ini menguraikan landasan konseptual dan tinjauan penelitian terdahulu yang relevan. Bagian ini mencakup pembahasan mendalam mengenai Encrypted Traffic Analysis (ETA), prinsip-prinsip klasifikasi berbasis aliran (flow-based), dan evolusi teknik klasifikasi lalu lintas VPN. Tinjauan kritis terhadap metode-metode yang ada (seperti pendekatan statistik, deep learning, dan representasi gambar ) dilakukan untuk memvalidasi kesenjangan penelitian yang diidentifikasi, sehingga memperkuat posisi dan kebaruan dari metodologi PESV yang diusulkan.
    
    \item \textbf{Bab 3: Metodologi Penelitian.} 
    Bab ini merupakan inti teknis yang menjelaskan secara detail alur kerja penelitian. Pembahasan dimulai dari pemilihan dataset ISCX 2016 dan tahap pra-pemrosesan data secara rigid, termasuk konversi paket-ke-aliran menggunakan SplitCap, verifikasi TCP handshake, dan pemfilteran noise, yang menghasilkan 8.763 aliran data bersih. Bagian utama bab ini memaparkan perancangan dan implementasi kerangka kerja baru Path-Embedding Signature Vector (PESV) $\Sigma=(\alpha,\beta,\gamma)$, dengan merinci implementasi setiap komponen: $\alpha$ (autoencoder pada urutan ukuran paket), $\beta$ (Wasserstein distance pada distribusi IAT), dan $\gamma$ (cosine similarity pada statistik burst). Bab ini ditutup dengan penjelasan tentang pembuatan basis data fitur (CSV) dan arsitektur model machine learning yang digunakan untuk klasifikasi.
    
    \item \textbf{Bab 4: Hasil dan Pembahasan.} 
    Bab ini menyajikan temuan empiris dari implementasi metodologi yang telah dirancang. Bagian ini akan memaparkan secara rinci skenario pengujian dan metrik evaluasi yang digunakan. Hasil kinerja model machine learning yang dilatih menggunakan basis data PESV  akan disajikan secara kuantitatif, mencakup performa untuk tugas klasifikasi aplikasi dan klasifikasi kategori. Bagian pembahasan akan menganalisis secara mendalam temuan tersebut, menginterpretasi kemampuan model, dan membandingkan efektivitas kerangka kerja PESV dalam memberikan visibilitas terhadap lalu lintas terenkripsi dibandingkan dengan pendekatan lainnya.
    
    \item \textbf{Bab 5: Penutup.} 
    Bab terakhir ini merangkum keseluruhan hasil penelitian. Bagian ini berisi kesimpulan yang secara lugas menjawab setiap butir rumusan masalah yang diajukan di Bab 1. Selain itu, bab ini juga menyajikan saran untuk penelitian di masa mendatang, seperti potensi penerapan metodologi PESV pada dataset yang lebih baru, eksplorasi arsitektur deep learning yang berbeda, atau pengembangan metode untuk aplikasi real-time.
\end{enumerate}